% ****** Start of file apssamp.tex ******
%
%   This file is part of the APS files in the REVTeX 4.1 distribution.
%   Version 4.1r of REVTeX, August 2010
%
%   Copyright (c) 2009, 2010 The American Physical Society.
%
%   See the REVTeX 4 README file for restrictions and more information.
%
% TeX'ing this file requires that you have AMS-LaTeX 2.0 installed
% as well as the rest of the prerequisites for REVTeX 4.1
%
% See the REVTeX 4 README file
% It also requires running BibTeX. The commands are as follows:
%
%  1)  latex apssamp.tex
%  2)  bibtex apssamp
%  3)  latex apssamp.tex
%  4)  latex apssamp.tex
%

\documentclass[%
 reprint,
 nobalance,
%superscriptaddress,
%groupedaddress,
%unsortedaddress,
%runinaddress,
%frontmatterverbose, 
%preprint,
%showpacs,preprintnumbers,
%nofootinbib,
%nobibnotes,
%bibnotes,
 amsmath,amssymb,
 aps,
%pra,
%prb,
%rmp,
%prstab,
%prstper,
%floatfix,
]{revtex4-1}

\usepackage{graphicx}% Include figure files
\usepackage{dcolumn}% Align table columns on decimal point
\usepackage{hyperref}% add hypertext capabilities
\usepackage{url}% url links
\usepackage{bm}% bold math
\usepackage{booktabs}% tables
\usepackage{listings}% codelisting
\usepackage[labelformat=parens,labelsep=quad,skip=3pt]{caption}% caption plots
\usepackage{subcaption}% subplots 
\usepackage{blindtext}% lorem ipsum...
%\usepackage[mathlines]{lineno}% Enable numbering of text and display math
%\linenumbers\relax % Commence numbering lines

%\usepackage[showframe,%Uncomment any one of the following lines to test 
%%scale=0.7, marginratio={1:1, 2:3}, ignoreall,% default settings
%%text={7in,10in},centering,
%%margin=1.5in,
%%total={6.5in,8.75in}, top=1.2in, left=0.9in, includefoot,
%%height=10in,a5paper,hmargin={3cm,0.8in},
%]{geometry}

\newcommand{\commandname}{\command}


\begin{document}

%\preprint{APS/123-QED}

\title{Project 3 - FYS3150}% Force line breaks with \\
\thanks{Computational Physics, autumn 2016, University of Oslo}%

\author{Andreas G. Lefdalsnes}
 % \altaffiliation[Also at ]{Physics Department, XYZ University.}%Lines break automatically or can be forced with \\

% \author{Second Author}%
%  \email{Second.Author@institution.edu}
\affiliation{%
 Student: University of Oslo, Department of Physics\\
 email-address: andregl@student.matnat.uio.no
}%

\author{Tellef Storebakken}
\affiliation{Student: University of Oslo, Department of Physics\\
 email-address: tellefs@student.matnat.uio.no}

% \collaboration{MUSO Collaboration}%\noaffiliation

% \author{Charlie Author}
%  \homepage{http://www.Second.institution.edu/~Charlie.Author}
% \affiliation{
%  Second institution and/or address\\
%  This line break forced% with \\
% }%
% \affiliation{
%  Third institution, the second for Charlie Author
% }%
% \author{Delta Author}
% \affiliation{%
%  Authors' institution and/or address\\
%  This line break forced with \textbackslash\textbackslash
% }%

% \collaboration{CLEO Collaboration}%\noaffiliation

\date{\today}% It is always \today, today,
             %  but any date may be explicitly specified

\begin{abstract}
An abstract

% \begin{description}
% \item[Usage]
% Secondary publications and information retrieval purposes.
% \item[PACS numbers]
% May be entered using the \verb+\pacs{#1}+ command.
% \item[Structure]
% You may use the \texttt{description} environment to structure your abstract;
% use the optional argument of the \verb+\item+ command to give the category of each item. 
% \end{description}

\end{abstract}

% % \pacs{Valid PACS appear here}% PACS, the Physics and Astronomy
% %                              % Classification Scheme.
% % %\keywords{Suggested keywords}%Use showkeys class option if keyword
% %                               %display desired

\maketitle

\section{\label{sec:Int}Introduction}
An introduction.


\section{\label{sec:The}Theory and methods}

\subsection{\label{sec:New}Newton's law of gravitation}
Newton's law of gravitation states for two objects of mass $m_1$ and $m_2$, the force on object 1 from object 2 is given by \footnote{All theory in this project adapted from FYS3150 Project 3 (Fall 2016) $\href{link}{linkname}$}

\begin{equation}
	\bm{F} = \frac{Gm_1 m_2}{r^2} \bm{u_r} = \frac{Gm_1 m_2}{r^3} \bm{r}
\end{equation}

where $G$ is the gravitational constant and $\bm{r}$ is a radial vector pointing at object 2. $r = \left|\bm{r}\right| $ is the distance and $\bm{u_r} = \bm{r}/r$ is a radial unit vector.


% \begin{widetext}	% Write a matrix of some kind
% \begin{equation}
% \bm{Au} =
% \begin{pmatrix}
%   \frac{2}{\hbar^{2}} + V_{1} & -\frac{1}{\hbar^{2}} & 0 & \cdots & \cdots & 0 \\
%   -\frac{1}{\hbar^{2}} & \frac{2}{\hbar^{2}} + V_{2} &  -\frac{1}{\hbar^{2}} & 0 &\cdots & \cdots \\
%   0 & -\frac{1}{\hbar^{2}} & \frac{2}{\hbar^{2}} + V_{3} & -\frac{1}{\hbar^{2}} & 0 & \cdots \\
%   \vdots & \vdots & \vdots & \vdots & \vdots & \vdots \\
%   0 & \cdots & \cdots & -\frac{1}{\hbar^{2}} & \frac{2}{\hbar^{2}} + V_{N-3} & -\frac{1}{\hbar^{2}} \\
%   0 & \cdots & \cdots & \cdots & -\frac{1}{\hbar^{2}} & \frac{2}{\hbar^{2}} + V_{N-2}
% \end{pmatrix}
% \begin{pmatrix}
% 	u_{1} \\
% 	u_{2} \\
% 	\vdots \\
% 	\\
% 	\\
% 	u_{N-2}
% \end{pmatrix}
% =	\lambda
% \begin{pmatrix}
% 	u_{1} \\
% 	u_{2} \\
% 	\vdots \\
% 	\\
% 	\\
% 	u_{N-2}
% \end{pmatrix}
% \end{equation}
% \end{widetext}

\subsection{\label{sec:Sec}The second part}

\begin{equation}\label{eq:22} % label an equation
	1 = 1
\end{equation}

\subsection{\label{sec:Thi}The third part}
A reference %\footnote{See M.H. Jensen, Computational Physics: Lecture Notes Fall 2015, ch. x.y, available at $\href{link}{linkname}$}

% \begin{equation} % another matrix
% S = 
% 	\begin{pmatrix}
% 		1 & 0 & \cdots & 0 & 0 \cdots & 0 & 0 \\
% 		0 & 1 & \cdots & 0 & 0 \cdots & 0 & 0 \\
% 		\cdots & \cdots & \cdots & \cdots & \cdots & 0 & \cdots \\
% 		0 & 0 & \cdots & \cos(\theta) & 0 \cdots & 0 & \sin(\theta) \\
% 		0 & 0 & \cdots & 0 & 1 \cdots & 0 & 0 \\
% 		\cdots & \cdots & \cdots & \cdots & \cdots & 0 & \cdots \\
% 		0 & 0 & \cdots & 0 & 0 \cdots & 1& 0 \\
% 		0 & 0 & \cdots & -\sin(\theta) & 0 \cdots & 0 & \cos(\theta) \\	
% 	\end{pmatrix}
% \end{equation}


\section{Results and discussion}

\subsection{\label{sec:Sub1}First subpart}
An equation reference % \eqref{eq:x}

% \begin{table}[h]	% a table
% \centering
% \caption{Caption}
% \label{my-label}	% necessary?
% \begin{tabular}{|l|l|}
% \hline
% \textbf{Method}  & \textbf{Time {[}s{]}} \\ \hline
% Jacobi algorithm & $13.73$               \\ \hline
% Armadillo        & $0.02$                \\ \hline
% \end{tabular}
% \end{table}


\subsection{\label{sec:Sub2}Second subpart}
More text.

% \begin{figure*}[t!]	% a big figure
%     \centering
%     \begin{subfigure}[t]{0.5\textwidth}
%         \centering
%         \includegraphics[height=3.2in]{../omega001.png}
%         \caption{Wavefunction for $\omega_{r} = 0.01$}
%     \end{subfigure}%
%     ~ 
%     \begin{subfigure}[t]{0.5\textwidth}
%         \centering
%         \includegraphics[height=3.2in]{../omega05.png}
%         \caption{Wavefunction for $\omega_{r} = 0.5$}
%     \end{subfigure}

%     \begin{subfigure}[t]{0.5\textwidth}
%         \centering
%         \includegraphics[height=3.2in]{../omega1.png}
%         \caption{Wavefunction for $\omega_{r} = 1$}
%     \end{subfigure}%
%     ~ 
%     \begin{subfigure}[t]{0.5\textwidth}
%         \centering
%         \includegraphics[height=3.2in]{../omega5.png}
%         \caption{Wavefunction for $\omega_{r} = 5$}
%     \end{subfigure}
%     \caption{Two-electron wavefunction for various values of $\omega_{R}$}
% \end{figure*}


\section{Conclusion}
Do stuff.

\section{Appendix}
All code used is available at: %\href{link}{linkname} \\
The programs used in this project are listed in this section:

\begin{description}
\item [main.cpp] Program1
\item [plot.py] Program2
\end{description}

%\tableofcontents 

\end{document}


% \section{\label{sec:level1}First-level heading}

% This sample document demonstrates proper use of REV\TeX~4.1 (and
% \LaTeXe) in mansucripts prepared for submission to APS
% journals. Further information can be found in the REV\TeX~4.1
% documentation included in the distribution or available at
% \url{http://authors.aps.org/revtex4/}.

% When commands are referred to in this example file, they are always
% shown with their required arguments, using normal \TeX{} format. In
% this format, \verb+#1+, \verb+#2+, etc. stand for required
% author-supplied arguments to commands. For example, in
% \verb+\section{#1}+ the \verb+#1+ stands for the title text of the
% author's section heading, and in \verb+\title{#1}+ the \verb+#1+
% stands for the title text of the paper.

% Line breaks in section headings at all levels can be introduced using
% \textbackslash\textbackslash. A blank input line tells \TeX\ that the
% paragraph has ended. Note that top-level section headings are
% automatically uppercased. If a specific letter or word should appear in
% lowercase instead, you must escape it using \verb+\lowercase{#1}+ as
% in the word ``via'' above.

% \subsection{\label{sec:level2}Second-level heading: Formatting}

% This file may be formatted in either the \texttt{preprint} or
% \texttt{reprint} style. \texttt{reprint} format mimics final journal output. 
% Either format may be used for submission purposes. \texttt{letter} sized paper should
% be used when submitting to APS journals.

% \subsubsection{Wide text (A level-3 head)}
% The \texttt{widetext} environment will make the text the width of the
% full page, as on page~\pageref{eq:wideeq}. (Note the use the
% \verb+\pageref{#1}+ command to refer to the page number.) 
% \paragraph{Note (Fourth-level head is run in)}
% The width-changing commands only take effect in two-column formatting. 
% There is no effect if text is in a single column.

% \subsection{\label{sec:citeref}Citations and References}
% A citation in text uses the command \verb+\cite{#1}+ or
% \verb+\onlinecite{#1}+ and refers to an entry in the bibliography. 
% An entry in the bibliography is a reference to another document.

% \subsubsection{Citations}
% Because REV\TeX\ uses the \verb+natbib+ package of Patrick Daly, 
% the entire repertoire of commands in that package are available for your document;
% see the \verb+natbib+ documentation for further details. Please note that
% REV\TeX\ requires version 8.31a or later of \verb+natbib+.

% \paragraph{Syntax}
% The argument of \verb+\cite+ may be a single \emph{key}, 
% or may consist of a comma-separated list of keys.
% The citation \emph{key} may contain 
% letters, numbers, the dash (-) character, or the period (.) character. 
% New with natbib 8.3 is an extension to the syntax that allows for 
% a star (*) form and two optional arguments on the citation key itself.
% The syntax of the \verb+\cite+ command is thus (informally stated)
% \begin{quotation}\flushleft\leftskip1em
% \verb+\cite+ \verb+{+ \emph{key} \verb+}+, or\\
% \verb+\cite+ \verb+{+ \emph{optarg+key} \verb+}+, or\\
% \verb+\cite+ \verb+{+ \emph{optarg+key} \verb+,+ \emph{optarg+key}\ldots \verb+}+,
% \end{quotation}\noindent
% where \emph{optarg+key} signifies 
% \begin{quotation}\flushleft\leftskip1em
% \emph{key}, or\\
% \texttt{*}\emph{key}, or\\
% \texttt{[}\emph{pre}\texttt{]}\emph{key}, or\\
% \texttt{[}\emph{pre}\texttt{]}\texttt{[}\emph{post}\texttt{]}\emph{key}, or even\\
% \texttt{*}\texttt{[}\emph{pre}\texttt{]}\texttt{[}\emph{post}\texttt{]}\emph{key}.
% \end{quotation}\noindent
% where \emph{pre} and \emph{post} is whatever text you wish to place 
% at the beginning and end, respectively, of the bibliographic reference
% (see Ref.~[\onlinecite{witten2001}] and the two under Ref.~[\onlinecite{feyn54}]).
% (Keep in mind that no automatic space or punctuation is applied.)
% It is highly recommended that you put the entire \emph{pre} or \emph{post} portion 
% within its own set of braces, for example: 
% \verb+\cite+ \verb+{+ \texttt{[} \verb+{+\emph{text}\verb+}+\texttt{]}\emph{key}\verb+}+.
% The extra set of braces will keep \LaTeX\ out of trouble if your \emph{text} contains the comma (,) character.

% The star (*) modifier to the \emph{key} signifies that the reference is to be 
% merged with the previous reference into a single bibliographic entry, 
% a common idiom in APS and AIP articles (see below, Ref.~[\onlinecite{epr}]). 
% When references are merged in this way, they are separated by a semicolon instead of 
% the period (full stop) that would otherwise appear.

% \paragraph{Eliding repeated information}
% When a reference is merged, some of its fields may be elided: for example, 
% when the author matches that of the previous reference, it is omitted. 
% If both author and journal match, both are omitted.
% If the journal matches, but the author does not, the journal is replaced by \emph{ibid.},
% as exemplified by Ref.~[\onlinecite{epr}]. 
% These rules embody common editorial practice in APS and AIP journals and will only
% be in effect if the markup features of the APS and AIP Bib\TeX\ styles is employed.

% \paragraph{The options of the cite command itself}
% Please note that optional arguments to the \emph{key} change the reference in the bibliography, 
% not the citation in the body of the document. 
% For the latter, use the optional arguments of the \verb+\cite+ command itself:
% \verb+\cite+ \texttt{*}\allowbreak
% \texttt{[}\emph{pre-cite}\texttt{]}\allowbreak
% \texttt{[}\emph{post-cite}\texttt{]}\allowbreak
% \verb+{+\emph{key-list}\verb+}+.
%
% ****** End of file apssamp.tex ******
